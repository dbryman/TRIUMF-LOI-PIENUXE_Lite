The support of proposal development and small prototypes is available among the research groups in our collaboration.  Our Japanese collaborators have received approximately \$100\,k
seed funding for developing an improved $R_{e/\mu}$ measurement and our UCSC group has support for LGAD development at the level we need at the moment.
Major funding for \nexp will be required for a new custom beamline at TRIUMF and for the detector and instrumentation package.
Following a successful experimental proposal defense, we  will approach NSERC to seek funds for a new low-energy pion/muon beamline for fundamental physics.
As advised by the TRIUMF Pre-Proposal Review Report (Feb 19, 2021), the new {\tt M13'} beam would cost $\approx 8-10$\,M\,CDN.

Detectors, electronics, and mechanical structures can be expected to cost as much or more than the beamline.  Given the anticipated costs of either a LXe or LYSO based calorimeter together with a finely instrumented Active Target, the appropriate U.S. funding possibilities include the NSF Mid-scale RI-1 program, which supports implementation projects in the cost range from \$6-20\,M.  Our collaboration has had several successful NSF MRI awards in the few \$M range that have delivered the custom instrumentation for the MuLan measurement of the muon lifetime and the Muon $g-2$ measurement of the muon magnetic anomaly.  The NSF is neatly cross-disciplinary and thus a nuclear / particle physics proposal can be successfully supported through the Mid-scale program.
