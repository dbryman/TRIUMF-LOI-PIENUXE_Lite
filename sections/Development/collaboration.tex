The developing \nexp Collaboration is drawn from physicists having considerable experience in experimental precision physics, together with leading theorists who are carefully articulating the needed goals to maximize scientific impact.
%All of the experimentalists have significant experience in balancing statistical and systematic uncertainties, in careful modeling of the apparatus, and in the design of custom instrumentation.
%
Member groups led previous generation rare pion decay experiments -- {\tt PIENU, PEN} and {\tt PiBeta} -- and their collective experience is central to the design decisions that guide our emerging plans for a next-generation rare-pion-decay facility.
Key to this LOI and to the months of working meetings held to date, are ``Lessons Learned," from the TRIUMF and PSI experiences.
Other collaborators have worked on rare kaon decay experiments ({\tt BNL-E787/E949}) and on low-energy stopped muon experiments ({\tt MEG, MuLan, MuCap} and {\tt MuSun}); all have similar stringent beam and detector performance demands.
%
Many members of our team are involved in the Fermilab Muon $g-2$ experimental campaign, which has demonstrated the impact that can be made by using next-generation instrumentation and modeling tools to control systematics and acquire data at high rates.
Our collaboratioin has built both crystal and LXe calorimeters, various TPCs, a vast array of fast scintillating counters, custom arrays of waveform digitizers, fast DAQ systems, and beamline instrumentation.
Especially relevant are the experiences from our Japanese collaborators who developed the unique LXe calorimeter for {\tt MEG} and our UCSC collaborators who are  world leaders in LGAD development.