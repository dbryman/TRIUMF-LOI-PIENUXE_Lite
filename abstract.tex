\abstract{

A next-generation rare pion decay experiment is strongly motivated by several inconsistencies between Standard Model (SM) predictions and data pointing towards the violation of lepton flavor universality. It can probe non-SM explanations of these anomalies due to sensitivity to quantum effects of new particles even if their masses are at a very high scale.
Using state-of-the-art instrumentation, computational resources, and a new high-intensity beam, the experiments can be performed at TRIUMF.
%with nearly the same apparatus and beamline. 
Measurement of the charged-pion branching ratio to electrons vs.\ muons $R_{e/\mu}$ is extremely sensitive to a wide variety of new physics effects. At present, the SM prediction for $R_{e/\mu}$ is known to 2 parts in $10^4$, which is 15 times more precise than the current experimental result.  An experiment at a comparable level of accuracy allows a  test of lepton flavor universality at an unprecedented level, probing mass scales up to 3000 TeV. Measurement of  the rare process of pion beta decay, $\pi^+\to \pi^0 e^+ \nu (\gamma)$) with an order of magnitude improvement in sensitivity will determine $V_{ud}$ in a theoretically pristine manner and test CKM unitarity at the quantum loop level. In addition, various exotic rare decays involving e.g.\ sterile neutrinos and  axions will be searched for with unprecedented sensitivity.  The experiment design benefits from experience with the recent PIENU and PEN efforts at TRIUMF and PSI.  Improved resolution, greatly increased calorimeter depth, high-speed detector and electronic response, large solid angle coverage, and complete event reconstruction   are all critical to the new approach which includes  a 3$\pi$ sr 30 radiation length  calorimeter, a segmented active stopping target, and an electron tracker, as well as a new customized pion production target and pion beam line.   }